\part{Jogando}
\chapter{Combate}
\section{Turnos}
%
%
Cada turno é dividido em duas fases ação e cálculo.
Na fase de ação cada ator em uma ordem especifica coloca ações para executar.
Na fase de cálculo as ações ocorrem em ordem inversa que foram colocadas.
%
\subsection{Fase de ação}
Começando pelo mais lento, cada ator pode gastar pontos de ação para colocar uma ação.
Toda vez que uma ação é colocada a vez passa denovo por todos os atores começando pelo mais lento,
após passar por um certo ator a reação só volta se alguém colocar uma ação.
não existem reações, uma reação é uma ação normal que tem como requerimento uma ação na pilha.
Um personagem sem ações para o limite por turno não pode agir, mas é possível um sem pontos de ação reagir com uma ação grátis.
Um personagem pode escolher agir com uma ação impossível,
Se na fase de cálculo não vier a ficar possível a ação ela é ignorada.
O mesmo ocorre para ações que eram possíveis mas ficaram impossíveis.

Um personagem pode gastar até o maior entre 3 ou um valor igual a sua adrenalina de estamina para ganhar 1 ponto de ação por estamina gasta.
esse ponto de ação é perdido no momento que começa o próximo turno.
\subsection{Fase de Cálculo}
Na fase de cálculo cada ação é executada do topo da pilha(mais recente).
%
%
\section{Ações Padrão}
ações que qualquer personagem pode fazer mesmo sem items ou equipamento
%
\subsection{Atacar Corpo a Corpo}
\paragraph{custo} 3pa
\paragraph{ao executar}
\begin{itemize}
  \item EA:avançar 1
  \item então se puder alcançar seu alvo role Artes marciais para acertar
  \item seu dano é metade de seu dano de força
\end{itemize}
%
\subsection{Interagir}
\paragraph{custo}
\begin{itemize}
  \item 1pa por segundo que necessita para fazer a ação
  \item mínimo de 1 pa
\end{itemize}
\paragraph{ao executar}
\begin{itemize}
  \item EA:caminhar 2 
  \item então faça o teste como normal
\end{itemize}
%
\subsection{Desviar}
\paragraph{custo} 2pa
\paragraph{ao executar}
\begin{itemize}
  \item EA:caminhar 2
  \item para acertar é necessário um sucesso maior ou igual ao seu
\end{itemize}
%
\subsection{Desviar com tudo}
\paragraph{custo} 3pa
\paragraph{requerimento} estar de pé 
\paragraph{ao executar} 
\begin{itemize}
  \item o próximo ataque que mira em voce é feito com 4 dados de penalidade se tiver sucesso, 
    6 se tiver sucesso grande e 8 com um sucesso extremo. sucesso crítico adiciona 10 dados % TODO: isso parece generalizável
  \item voce cai no chão
\end{itemize}
%
\subsection{Movimentar}
\paragraph{custo} 1pa
\paragraph{ao executar} EA:caminhar 5

%
%
%
%
\chapter{Testes}


