\documentclass[portuguese]{book}
\usepackage{xcolor}
\usepackage{hyperref}
\hypersetup{
    colorlinks,
    citecolor=black,
    filecolor=black,
    linkcolor=black,
    urlcolor=black,
    linktoc=all
}
\usepackage{import}
\usepackage{babel}
\usepackage{graphicx}
\usepackage{geometry}
\title{Fisica Aurora Manual de Regras}
\date{\today}
\author{eldritch cookie}
%TODO: usar \subimport do pacote import quando tamanho crescer
\begin{document}
\pagecolor{gray}
  \maketitle
  \newpage
  \tableofcontents
  \import{./}{introduction}
  \import{./}{character}
  \import{./}{assets}
  \import{./}{playing}
  \import{./}{adventures}
  \import{./}{bestiary}
  \import{./}{appendix}

\end{document}
% ideias matemática deveria funcionar para ambos nível baixos e níveis altos
% pessoa normal 100 a 150 pontos de personagem
% personagens iniciais 200 pontos
% máximo 1000 pontos deveriam ser os deuses mais fortes de um panteão
% 20-60 pontos inciais em perícias mentais 20-60 em perícias físicas, 40-120 em estatisticas 20-60 em vantagens
% 12 perícias para um personagem seria 530 para nível 45 e 720 para 60 assumindo 3 perícias compartilhadas
% isso seria 36 perícias total * 3 108
% personagem de 150 sem vantagens
% 70 em estatisticas
% adrenalina 3(30pp) + qualidade física 3(18pp) agilidade 2(6pp) força 2(6pp) constituição 2(6pp) + 4pp sobrando
% 40 perícias mentais
% 40 vida 40 estamina 5 dano
% 40 perícias físicas 200
% x(x+1)*5 + x(C - x(x+1)/2)
% 5x^2+ 5x +xC -x^3/2-x^2/2 => 10x+ 5 + C - 3x^2/2 -x = 0 
% x =  (3⁺+sqrt(81 - 4(-1.5)(5+C)/3)
% x = 3 + sqrt(111+6C)/3
% níveis
% roladas normais difíceis
% roladas normais moderada
% roladas normais fáceis
% roladas normais triviais roladas grandes moderadas 
% roladas grandes fáceis
% roladas grandes triviais roladas extremas moderadas 
% roladas extremas fáceis 
% roladas extremas triviais 
% para cada nível é relevante o número e proporção de perícias no nível deuses de 1000 pontos teriam todas as perícias em 500+ 
