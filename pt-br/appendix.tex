\part{Apêndice}
\appendix
\chapter{Glossário}
\section{x/nível/nível}
valor cumulativo com crescimento por nível igual ao nível vezes x.
após segue um exemplo para 2/nível/nível

\begin{tabular}[c]{c|c|c}
\hline
  nível & aumento & valor total  \\
\hline
  1   & 2   & 2   \\
  2   & 4   & 6   \\
  3   & 6   & 12  \\
  4   & 8   & 20  \\
  5   & 10  & 30  \\
  ... & ... & ... \\
  10  & 20  & 110 \\ 
  ... & ... & ... \\
  20  & 40  & 420 \\
  ... & ... & ... \\
\hline
\end{tabular}

\chapter{Tabelas}
\chapter{Matemática}
\section{1/nível/nível}
\paragraph{fórmula} \[n(n+1)/2\]
\paragraph{resolução} para chegar a resolução é simples nós temos 1+2+3+...+n
se pegarmos o dobro nós podemos fazer um truque para conseguir somas de n+1 basta somar em direção crescente e descrescente simultanêamente.
\[(1+n) + (2 + (n-1)) + (3 + (n-2))+ ... +((n-1)+2) + (n+1) \]
quantos \[n+1\] temos? bem nós começamos com o 1 e terminamos com o n
então são n \[n+1\] como isso é o dobro do valor que queremos precisamos dividir por 2 chegando na fórmula final 
\chapter{Índices}
TO DO: fazer indices
\section{Índice de Perícias}
\section{Índice de Técnicas}
\section{Índice de Vantagens}
\section{Índice de Desvantagens}
\section{Índice de Items}
\section{Índice de Ações}
\section{Índice de Condições}
\section{Índice de Monstros}
