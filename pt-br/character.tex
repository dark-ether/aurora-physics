\part{Personagens}
\chapter{Estatísticas}
\section{Adrenalina}
% TODO:inserir descrição
\paragraph{custo} 5pp/nível/nível
\paragraph{efeitos por nível}
\begin{itemize}
  \item aumenta seus pontos de ação naturais em 1
  \item aumenta seu limite de ações natural por turno em 1
  \item aumenta seus pontos de risco em 1
\end{itemize}
%
%
\section{Metacognição}
% TODO:inserir descrição
\paragraph{custo} 1pp/nível/nível
\paragraph{efeito} fornece 10 pontos de conhecimento/nível/nível
%
%
\section{Aprendizado}
% TODO:inserir descrição
\paragraph{custo} 1pp/nível
\paragraph{efeito por nível} fornece 1 ponto de conhecimento por nível de metacognição
%
%
\section{Qualidade Física}
% TODO:inserir descrição
\paragraph{custo} 3pp/nível/nível
\paragraph{efeitos por nível}
\begin{itemize}
    \item aumenta sua vida máxima natural após modificadores em 5
    \item aumenta sua estamina máxima natural após modificadores em 5
    \item aumenta sua força, constituição e agilidade natural antes de modificadores em 1
\end{itemize}
%
%
\section{Agilidade}
% TODO:inserir descrição
\paragraph{custo} 2pp/nível/nível
\paragraph{efeitos por nível}
\begin{itemize}
  \item aumenta sua estamina máxima natural em 3 após modificadores
  \item a cada 5 níveis aumenta seus pontos de ação naturais antes de modificadores em 1
\end{itemize}
%
%
\section{Constituição}
% TODO:inserir descrição
\paragraph{custo} 2pp/nível/nível
\paragraph{efeitos por nível}
\begin{itemize}
  \item aumenta sua vida máxima após modificadores em 5
  \item a cada 5 níveis diminui dano recebido por ataques físicos em 1
\end{itemize}
%
%
\section{Força}
% TODO:inserir descrição
\paragraph{custo} 2pp/nível/nível
\paragraph{efeitos por nível} 
\begin{itemize}
  \item aumenta sua estamina máxima natural em 2 após modificadores
  \item a cada 3 níveis proporciona 1d6 de dano de força antes de modificadores
  \item cada nível restante proporciona 1 de dano de força antes de modificadores
\end{itemize}
\paragraph{exemplo} um personagem com 3 de qualidade física e 2 de força teria 1d6+2 dano de força 
%
%
\section{Vida}
% TODO:inserir descrição
\paragraph{propriedade} recurso associado
\paragraph{custo} 1pp/nível
%
%
\section{Estamina}
% TODO:inserir descrição
\paragraph{propriedade} recurso associado
\paragraph{custo} 1pp/nível
%
%
\section{Mana}
% TODO:inserir descrição
\paragraph{propriedade} recurso associado
\paragraph{custo} 2pp/nível
\section{Biotipo}
Seu biotipo pode ser um entre os mencionados nessa seção, é obrigatório escolher um.
\subsection{Ágil}
\paragraph{efeitos}
\begin{itemize}
  \item por 1pp voce pode ganhar um número de pontos de treinamento físico 
    igual a sua agilidade natural
\end{itemize}
\subsection{Forte}
\paragraph{efeitos}
\begin{itemize}
  \item por 1pp voce pode ganhar um número de pontos de treinamento físico 
    igual a sua força natural
\end{itemize}
\subsection{Resiliente}
\paragraph{efeitos}
\begin{itemize}
  \item por 1pp voce pode ganhar um número de pontos de treinamento físico 
    igual a sua constituição natural
\end{itemize}
%
%
%
%
% TODO: separar para próprio arquivo
\chapter{Perícias}
Perícias são as estatisticas testáveis elas são compradas ou com pontos de treinamento físico ou com pontos de conhecimento.
Ao se fazer um teste se rola 1 d100 e compara o valor rolado com o valor da perícia.
Se o resultado do dado após modificadores for menor ou igual ao valor da perícia o teste é um sucesso.
Se for menor ou igual a metade é um sucesso grande.
Se for menor ou igual a um quinto é um sucesso extremo.
Se for menor ou igual a um décimo é um sucesso crítico.
Se for rolado 1 natural no dado é automaticamente um sucesso e o grau do sucesso é crítico.
Se for rolado acima do valor da perícia no teste é uma falha.

Rolar acima de 80 natural em um teste de pericía causa uma falha automática e grande independente do valor da perícia.
Rolar 40 ou mais que sua perícia é falha grande.
Rolar 80 ou mais que sua perícia é falha extrema.
Rolar 160 ou mais que sua perícia é falha crítica.

cada perícia tem um pré-requisito para passar dos níveis 80, 160 e 400.
perícias tem fases 0 é não treinado, 1-80 é treinado, 100-160 é perito,
200-400 é mestre, 500+ lendário

ao chegar a 100, 200 e 500 se ganha uma vantagem especial que não pode ser comprada normalmente chamada de técnica.
% TODO: adicionar perícias para armas pouco usadas
% TODO: perícias de conhecimento plantas, animais,
% TODO: encher até 108 perícias
% por perícia
% se pode gastar pontos de conhecimento e ou pontos de treinamento físico
% requerimentos para 80, 160 e 400
% quando precisar usar a perícia
% quais situações dá penalidades e bonus
% vantagens relacionadas
% desvantagens relacionadas
% qual tipo de teste tem que ser feito
\section{Agarrar}
% TODO: como irá funcionar?
\paragraph{Propriedade} Física
\paragraph{Modificadores} 
\begin{itemize}
  \item um dado de penalidade perito se seu alvo tiver tamanho 2 maior que o seu 
  \item um dado de penalidade mestre se seu alvo tiver tamanho 3 maior que o seu 
  \item um dado de penalidade lendário por tamanho extra acima de 5 a mais que o seu 
\end{itemize}
Agarrar é a capacidade de usar sua força para restringir movimentos de outros
\subsection{Ação Agarrar}
\paragraph{custo} 1pa 

\section{Astronomia}
\section{Alquimia} Nigredo, Albedo, Cinitritas, Rubedo, Aurora
\section{Arco}
\section{Arremessar}
\section{Artes Marciais}
\section{Aterrissar}
\section{Atirar Pistola} tudo que usa uma mão e mira como uma arma de fogo% TODO: reavaliar
\section{Atirar Rifle} tudo que usa duas mãos e mira como uma arma de fogo% TODO: reavaliar
\section{Audição}
\section{Aumentar Moral}
\section{Barganhar}
\section{Caçar em Ambiente} por Bioma
\section{Canto}
\section{Chicote}
\section{Cirúrgia}
\section{Coordenar Grupo} % TODO: reavaliar
\section{Correr}
\section{Domar animais} Mamifero, Aves, Répteis
\section{Disfarçar}
\section{Dirigir}
\section{Discernir Mentira}
\section{Distrair} social
\section{Edição de midia digital} Video, Fotos
\section{Enganar}
\section{Escalar}
\section{Esconder Objeto}
\section{Esconder-se}
\section{Escultura}
\section{Esgueirar}
\section{Espada}
\section{Faca}
\section{Fotografia}
\section{Ginástica}
\section{Gamer} por jogo
\section{Gravar Video}
\section{Identificar} Plantas, Animais, Relíquias
\section{Lança}
\section{Machado}
\section{Nadar}
\section{Navegar}% TODO: reavaliar
\section{Olfato}
\section{Pesquisa} habilidade de achar material relevante não está relacionada a entendimento % TODO: reavaliar
\section{Pesca}
\section{Pilotar}
\section{Pintura} Digital, Manual
\section{Primeiros Socorrros}
\section{Pular}
\section{Reagir}
\section{Sobrevivência em Ambiente} por Bioma
\section{Tolerância Física} especial
\section{Tolerância Mental} especial
\section{Treinar Outros} por perícia física
\section{Tutoria} por perícia mental
\section{Visão}
%
%
%
%
%
%
%
%
%
%
\chapter{Carateristicas}
\section{Tamanho} 
0 é humano 1,5m multiplica ou divide por dois para aumentar ou diminuir o tamanho respectivamente
\chapter{Lista De Características}
Neste capítulo estão descritas várias características.
Requerimentos são os necessários para comprar certo nível.
Se não mencionar o requerimento não mencionar o nível,
o requerimento é para o primeiro nível.
\section{Legenda}
por x: 
  não é uma caracteristica e sim uma família de caracteristicas tendo uma diferente para cada x,
requerimento: significa requerimento para comprar
% vantagem barata 1-5 vantagem custo médio 5-10 custo alto 10-30
\section{A}
\section{B}
\section{C}
\subsection{Competência Confiável}
\paragraph{propriedade} por perícia
\paragraph{custo} 1pp/nível
\paragraph{requerimentos}
\begin{itemize}
  \item perito na perícia escolhida para nível 1 e acima.
  \item mestre na perícia escolhida para nível 11 e acima.
\end{itemize}
\paragraph{efeito por nível} aumenta o valor necessário no d100 para falha automática em 1
\section{D}
\section{E}
\subsection{Estatistica Robusta}
\paragraph{propriedade} por estatistica
\paragraph{requerimento} estatistica tem recurso associado
\paragraph{custo} 5pp/nível/nível
\paragraph{efeito} multiplica a estatistica escolhida por (1+nível da vantagem)
\section{F}
\section{G}
\section{H}
\section{I}
\section{J}
\section{J}
\section{K}
\section{L}
\section{M}
\section{N}
\section{O}
\section{P}
\section{Q}
\section{R}
\section{S}
\section{T}
\section{U}
\section{V}
\section{W}
\section{X}
\section{Y}
\section{Z}

